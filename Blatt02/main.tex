\documentclass[DIN, pagenumber=false, fontsize=11pt, parskip=half]{scrartcl}

\usepackage{amsmath}
\usepackage{amsfonts}
\usepackage{amssymb}
\usepackage{enumitem}
\usepackage[utf8]{inputenc} % this is needed for umlauts
\usepackage[ngerman]{babel} % this is needed for umlauts
\usepackage[T1]{fontenc} 
\usepackage{commath}
\usepackage{xcolor}

\title{Grundlagen der Betriebssysteme}
\author{Tim Luchterhand, Paul Nykiel (Gruppe 017)}

\begin{document}
    \maketitle
    \section{Festkomma Darstellung}
    \begin{enumerate}[label=(\alph*)]
        \item 
            \begin{equation*}
                7.75 = 4 + 2 + 1 + 0.5 + 0.25 = {00111110}_2
            \end{equation*}
        \item
            \begin{equation*}
                2.71 \approx 2 + 0.5 + 0.25 = {00010110}_2
            \end{equation*}
            Fehler: $\abs{2.71 - 2.75} = 0.04$
        \item
            \begin{equation*}
                5.375 = 4 + 1 + 0.25 + 0.125 = {00101011}_2
            \end{equation*}
        \item
            \begin{equation*}
                9.12 \approx 8 + 1 + 0.125 = {01001001}_2
            \end{equation*}
            Fehler : $\abs{9.12 - 9.125} = 0.005$
    \end{enumerate}
    \section{Gleitkomma Darstellung}
    Vorzeichen in \textcolor{red}{rot}, Exponent in \textcolor{green}{grün}, Mantisse in \textcolor{blue}{blau}.
    \begin{enumerate}[label = (\alph*)]
        \item
            \begin{equation*}
                17.75 = {(-1)}^0 \cdot (1+0.109375) \cdot 2^{131-127} = \textcolor{red}{0}\ \textcolor{green}{10000011}\ \textcolor{blue}{00011100000000000000000}
            \end{equation*}
        \item 
            \begin{equation*}
                3.625 = {(-1)}^0 \cdot (1+0,8125) \cdot 2^{128 - 127} = \textcolor{red}{0}\ \textcolor{green}{10000000}\ \textcolor{blue}{110100000000000000000000}
            \end{equation*}
    \end{enumerate}
    \section{Gleitkomma Interpretationen}
    \begin{enumerate}[label=(\alph*)]
        \item Addition:
            \begin{equation*}
                2^{128-127} \cdot 1.0010111 + 2^{128-127} \cdot 1.1001011 = 2^{129-127} \cdot 1.0110001
            \end{equation*}
            Multiplikation:
            \begin{equation*}
                2^{128-127} \cdot 1.0010111 \cdot 2^{128-127} \cdot 1.1001011 = 2^{129 - 127} \cdot 1.11011110111101
            \end{equation*}
            Division:
            \begin{equation*}
                2^{128-127} \cdot 1.0010111 / 2^{128-127} \cdot 1.1001011 = 2^{127 - 127} \cdot 1.000000011
            \end{equation*}
        \item Addition:
            \begin{eqnarray*}
                -1 \cdot 2^{131-127} \cdot 1.1011 + 2^{127-127} \cdot 1.000011 
                &\stackrel{\text{Auf größeren Exponenten normalisieren}}{=}&\\
                -1 \cdot 2^{131-127} \cdot 1.1011 + 2^{131-127} \cdot 0.0001000011 &=&\\
                -1 \cdot 2^{131-127} \cdot 1.11000001
            \end{eqnarray*}
            Multiplikation:
            \begin{eqnarray*}
                -1 \cdot 2^{131-127} \cdot 1.1011 \cdot 2^{127-127} \cdot 1.000011 &=&\\                 
                -1 \cdot 2^{131-127} \cdot 1.110001001
            \end{eqnarray*}
            Division:
            \begin{eqnarray*}
                -1 \cdot 2^{131-127} \cdot 1.1011 / 2^{127-127} \cdot 1.000011 &=&\\                 
                -1 \cdot 2^{131-127} \cdot 1.100111001
            \end{eqnarray*}
        \item Addition:
            \begin{eqnarray*}
                -1 \cdot 2^{128-127} \cdot 1.1010011 + -1 \cdot 2^{133-127} \cdot 1.0101011
                &\stackrel{\text{Auf größeren Exponenten normalisieren}}{=}&\\
                -1 \cdot 2^{133-27} \cdot 0.000011010011 + -1 \cdot 2^{133-127} \cdot 1.0101011 &=&\\
                -1 \cdot 2^{133-27} \cdot 1.01100011
            \end{eqnarray*}
            Multiplikation
            \begin{eqnarray*}
                -1 \cdot 2^{128-127} \cdot 1.1010011 \cdot -1 \cdot 2^{133-127} \cdot 1.0101011 &=&\\
                2^{135-127} \cdot 1.0001101
            \end{eqnarray*}
            Division:
            \begin{eqnarray*}
                -1 \cdot 2^{128-127} \cdot 1.1010011 / -1 \cdot 2^{133-127} \cdot 1.0101011 &=&\\
                2^{122} \cdot 1,001111
            \end{eqnarray*}
    \end{enumerate}
    \section{Bitinterpretation}
    \begin{enumerate}[label=(\alph*)]
        \item Vorzeichen in \textcolor{red}{rot}, Exponent in \textcolor{green}{grün}, Mantisse in \textcolor{blue}{blau}.
            \begin{eqnarray*}
                &&{4496A000}_{16} \\&=&
                0100\ 0100\ 1001\ 0110\ 1010\ 0000\ 0000\ 0000 \\&=& 
                \textcolor{red}{0}\ \textcolor{green}{10001001}\ \textcolor{blue}{00101101010000000000000} \\&=&
                {(-1)}^0 \cdot \left(1+\frac{1}{8}+\frac{1}{32}+\frac{1}{64}+\frac{1}{256}+\frac{1}{1024}\right) \cdot 2^{137-127} \\&=&
                {1205}_{10}
            \end{eqnarray*}
        \item
            Erste Zahl:
            \begin{eqnarray*}
                &&{4496}_{16} \\&=&
                4 \cdot 16^3 + 4 \cdot 16^2 + 9 \cdot 16^1 + 6 \cdot 16^0 \\&=&
                {17558}_{10}
            \end{eqnarray*}
            Zweite Zahl (wir betrachten die Zahl als vorzeichenlose Zahl):
            \begin{eqnarray*}
                &&{A000}_{16} \\&=&
                10 \cdot 16^3 + 0 \cdot 16^2 + 0 \cdot 16^1 + 0 \cdot 16^0 \\&=&
                {40960}_{10}
            \end{eqnarray*}
    \end{enumerate}
    \section{UTF8 Darstellung}
    Startsequenz in \textcolor{green}{grün}, Bytestartsequenz in \textcolor{red}{rot}, Daten in \textcolor{blue}{blau}.
    \begin{enumerate}[label=(\alph*)]
        \item
            \begin{equation*}
                \text{U+}202E = \textcolor{green}{1110} \ \textcolor{red}{\ 0010} \  \textcolor{blue}{10} \textcolor{red}{00\ 0000} \ \textcolor{blue}{10} \textcolor{red}{10\ 1110}
            \end{equation*}
        \item
            \begin{equation*}
                \textcolor{green}{1111\ 0} \textcolor{red}{000}\ \textcolor{blue}{10} \textcolor{red}{01\ 1111}\ \textcolor{blue}{10} \textcolor{red}{01\ 1000}\ \textcolor{blue}{10} \textcolor{red}{00\ 1000} = \text{U+}1F608
            \end{equation*}
    \end{enumerate}
\end{document}
