\documentclass[DIN, pagenumber=false, fontsize=11pt, parskip=half]{scrartcl}

\usepackage{amsmath}
\usepackage{amsfonts}
\usepackage{amssymb}
\usepackage{enumitem}
\usepackage[utf8]{inputenc} % this is needed for umlauts
\usepackage[ngerman]{babel} % this is needed for umlauts
\usepackage[T1]{fontenc} 
\usepackage{commath}
\usepackage{xcolor}
\usepackage{booktabs}
\usepackage{float}
\usepackage{tikz-timing}
\usepackage{tikz}

\title{Grundlagen der Betriebssysteme}
\author{Tim Luchterhand, Paul Nykiel (Gruppe 017)}

\begin{document}
    \maketitle
    \section{Journaling und Log-Structured File Systems}
    \begin{enumerate}[label=(\alph*)]
        \item 
            \begin{enumerate}
                \item Änderung an Block 2 in Log eintragen
                \item Änderung an Block 2 tatsächlich durchführen
                \item Änderung an Block 3 in Log eintragen
                \item Änderung an Block 3 tatsächlich durchführen
            \end{enumerate}
            Nach Abschluss der Änderung ist sowohl die eigentliche Änderung an Block 2 und 3, als auch ein korrespondierender Eintrag im Log sichtbar.
        \item 
            \begin{enumerate}
                \item Block 2 kopieren und Änderungen durchführen
                \item Refernenzen auf Block 2 in der Inode der Datei anpassen
                \item Inode Tabelle kopieren und Verweis auf kopierte Inode ändern
                \item Refernenz des Superblocks auf kopierte Inode-Tabelle ändern (atomar).
                \item Block 3 kopieren und Änderungen durchführen
                \item Refernenzen auf Block 3 in der Inode der Datei anpassen
                \item Inode Tabelle kopieren und Verweis auf kopierte Inode ändern
                \item Refernenz des Superblocks auf kopierte Inode-Tabelle ändern (atomar).
            \end{enumerate}
            Nach erfolgreichem Abschluss der Transaktion sind lediglich die Änderungen in der Datei sichtbar, es existiert kein Log.
        \item Bei einem Journaling FS kann es eventuell möglich sein, die Datei wiederherzustellen. Dies ist dann möglich, wenn aus dem Log-File ersichtlich ist was zu tun ist. Sonst wird in das Dateisystem durch einen Undo der letzten Transaktion in einen konsistenten Zustand gebracht. 
            
            Bei einem Log-Structured FS ist das Wiederherstellen nicht möglich, das FS ist aber garantiert in einem konsistenten Zustand.
    \end{enumerate}

    \section{Speicherlokalität}
    \begin{enumerate}[label=(\alph*)]
        \item $ $
            \begin{table}[H]
                \centering
                \begin{tabular}{ccccccccccccccccc} 
                    \toprule
                    Addresse & 0 & 1 & 2 & 3 & 4 & 5 & 6 & 7 & 8 & 9 & A & B & C & D & E & F\\
                    \midrule
                    Wert & 10 & 4 & 34 & 8 & 11 & 6 & 12 & A  & 21 & FF & 13 & C & 14 & FF \\
                    \bottomrule
                \end{tabular}
                \caption{Eine Liste der Länge 5, mit den Elementen 10 bis 14}
            \end{table}
            \begin{table}[H]
                \centering
                \begin{tabular}{ccccccccccccccccc} 
                    \toprule
                    Addresse & 0 & 1 & 2 & 3 & 4 & 5 & 6 & 7 & 8 & 9 & A & B & C & D & E & F\\
                    \midrule
                    Wert &  & & 34 & 8 & & & & & 21 & FF & 40 & 41 & 42 & 43 & 44 & 45\\
                    \bottomrule
                \end{tabular}
                \caption{Ein Array der Länge 6, mit den Werten 40 bis 45}
            \end{table}
            \begin{table}[H]
                \centering
                \begin{tabular}{ccccccccccccccccc} 
                    \toprule
                    Addresse & 0 & 1 & 2 & 3 & 4 & 5 & 6 & 7 & 8 & 9 & A & B & C & D & E & F\\
                    \midrule
                    Wert & 4  &  50 & 34 & 8 & A & 51 & 52& 53 & 21 & FF & FF & 54 & 55 & 56 \\
                    \bottomrule
                \end{tabular}
                \caption{Eine verlinkte Arrayliste der Länge 7, mit den Elementen 50 bis 56}
            \end{table}
        \item
            \begin{itemize}
                \item Liste: 5 Zugriffe
                \item Array: 2 Zugriffe
                \item Verlinkte-Arrayliste: 5 Zugriffe
            \end{itemize}
    \end{enumerate}
\end{document}
