\documentclass[DIN, pagenumber=false, fontsize=11pt, parskip=half]{scrartcl}

\usepackage{amsmath}
\usepackage{amsfonts}
\usepackage{amssymb}
\usepackage{enumitem}
\usepackage[utf8]{inputenc} % this is needed for umlauts
\usepackage[ngerman]{babel} % this is needed for umlauts
\usepackage[T1]{fontenc} 
\usepackage{commath}
\usepackage{xcolor}
\usepackage{booktabs}
\usepackage{float}

\title{Grundlagen der Betriebssysteme}
\author{Tim Luchterhand, Paul Nykiel (Gruppe 017)}

\begin{document}
    \maketitle
    \section{Befehlsabarbeitung}
    \begin{enumerate}[label=(\alph*)]
        \item 
            Der Prozessor wiederholt jeden Befehlszyklus folgende Aufgaben:
            \begin{enumerate}
                \item Lade Befehlsregister aus PC (in Instruktionsregister)
                \item Interpretiere den Befehl
                \item Führe den Befehl aus
                \item PC inkrementieren
            \end{enumerate}
        \item $ $
            \begin{table}[H]
                \centering
                \begin{tabular}{cccc}
                    \toprule
                    Befehl & $\text{R}_0$ & $\text{R}_1$ & PC \\
                    \midrule
                       & e6 & 04 & 00 \\
                    00 & a4 & 04 & 04 \\
                    04 & a4 & 04 & a4 \\
                    a4 & 02 & 04 & a8 \\
                    a8 & 02 & 02 & ac \\
                    ac & 02 & 02 & b0 \\
                    b0 & 02 & 02 & b4 \\
                    b4 & 02 & 02 & a8 \\
                    a8 & 02 & 00 & ac \\
                    ac & 02 & 00 & b0 \\
                    b0 & 02 & 00 & 08 \\
                    08 & 02 & 00 & 00 \\
                    0c & 02 & 00 &    \\
                    \bottomrule
                \end{tabular}
            \end{table}
    \end{enumerate}
    \section{Interrupts}
    \begin{enumerate}[label=(\alph*)]
        \item Es kommt ein Syscall vor (durch Aufrufen des Stop Befehls bzw. des Ausgabe-Befehls), um dem Betriebssystem zu signalisieren, dass das Programm fertig ist bzw. eine Ausgabe zu tätigen.
        \item Alle Register (auch PC und CCR) werden in Registern gesichert. Dann wird in den Kernel-Mode gewechselt und der PC für die ISR geladen.
        \item Vor ausführen des Interrupts muss das passende Interrupt-Flag gelöscht werden (Befehl ${5\text{C}}_{16}$), damit der Interrupt nach ausführen nicht sofort erneut ausgeführt wird. 
            
            Nach ausführen des Interrupts (Befehl ${64}_{16}$) muss der Befehl RTI aufgerufen werden, um den ursprünglichen Zustand wiederherzustellen.
        \item Der Interrupt vor ${\text{B}0}_{16}$ könnte einerseits ein externer Interrupt sein (ausgelöst durch Timer oder Peripherie), andererseits ein User-Interrupt (ausgelöst durch das Aufrufen des Ausgabe-Befehls).
        \item Außer Trap und externem Interrupt gibt es noch den internen Interrupt, der bei internen Fehlern/Exceptions (z.B. Division durch Null) auftritt.
    \end{enumerate}
\end{document}
