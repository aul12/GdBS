\documentclass[DIN, pagenumber=false, fontsize=11pt, parskip=half]{scrartcl}

\usepackage{amsmath}
\usepackage{amsfonts}
\usepackage{amssymb}
\usepackage{enumitem}
\usepackage[utf8]{inputenc} % this is needed for umlauts
\usepackage[ngerman]{babel} % this is needed for umlauts
\usepackage[T1]{fontenc} 
\usepackage{commath}
\usepackage{xcolor}
\usepackage{booktabs}
\usepackage{float}

\title{Grundlagen der Betriebssysteme}
\author{Tim Luchterhand, Paul Nykiel (Gruppe 017)}

\begin{document}
    \maketitle
    \section{Befehlsabarbeitung}
    \begin{enumerate}[label=(\alph*)]
        \item 
            Der Prozessor wiederholt jeden Befehlszyklus folgende Aufgaben:
            \begin{enumerate}
                \item Lade Befehlsregister aus PC (in Instruktionsregister)
                \item Interpretiere den Befehl
                \item Führe den Befehl aus
                \item PC inkrementieren
            \end{enumerate}
        \item $ $
            \begin{table}[H]
                \centering
                \begin{tabular}{cccc}
                    \toprule
                    Befehl & $\text{R}_0$ & $\text{R}_1$ & PC \\
                    \midrule
                       & e6 & 04 & 00 \\
                    00 & a4 & 04 & 04 \\
                    04 & a4 & 04 & a4 \\
                    a4 & 02 & 04 & a8 \\
                    a8 & 02 & 02 & ac \\
                    ac & 02 & 02 & b0 \\
                    b0 & 02 & 02 & b4 \\
                    b4 & 02 & 02 & a8 \\
                    a8 & 02 & 00 & ac \\
                    ac & 02 & 00 & b0 \\
                    b0 & 02 & 00 & 08 \\
                    08 & 02 & 00 & 00 \\
                    0c & 02 & 00 &    \\
                    \bottomrule
                \end{tabular}
            \end{table}
    \end{enumerate}
    \section{Interrupts}
    \begin{enumerate}[label=(\alph*)]
        \item Es kommt ein Trapp-Interrupt vor (durch Aufrufen des Stop Befehls), um dem Betriebssystem zu signalisieren, dass das Programm fertig ist.
        \item Der Wert von $\text{R}_1$ muss vor der Ausführung
            gespeichert werden und danach wieder nach $\text{R}_1$ 
            geladen werden.
        \item TODO
        \item TODO
    \end{enumerate}
\end{document}
