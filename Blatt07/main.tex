\documentclass[DIN, pagenumber=false, fontsize=11pt, parskip=half]{scrartcl}

\usepackage{amsmath}
\usepackage{amsfonts}
\usepackage{amssymb}
\usepackage{enumitem}
\usepackage[utf8]{inputenc} % this is needed for umlauts
\usepackage[ngerman]{babel} % this is needed for umlauts
\usepackage[T1]{fontenc} 
\usepackage{commath}
\usepackage{xcolor}
\usepackage{booktabs}
\usepackage{float}
\usepackage{tikz-timing}
\usepackage{tikz}

\title{Grundlagen der Betriebssysteme}
\author{Tim Luchterhand, Paul Nykiel (Gruppe 017)}

\begin{document}
    \maketitle
    \section{Deadlocks}
    \begin{enumerate}[label=(\alph*)]
        \item Jedes Fahrzeug fährt einen \glqq{}Schritt\grqq{} nach vorne, sodass sich alle Autos auf der ersten Spur befinden.
            Dann kann kein Auto weiterfahren, da jedes Auto jeweils blockiert wird.
        \item Nein, dieses Szenario gibt es nicht. Angenommen es gäbe diese Situation, dann könnten drei Autos die Kreuzung verlassen. Dadurch wird allerdings das vierte Auto nicht mehr blockiert.
        \item 
            \begin{eqnarray*}
                \text{Fahrzeug} &\Leftrightarrow& \text{Prozess} \\
                \text{Kreuzung} &\Leftrightarrow& \text{Ressource (z.B. CPU-Zeit)} \\
                \text{Warten} &\Leftrightarrow& \text{Blockiert} \\
                \text{Nie überqueren können} &\Leftrightarrow& \text{Deadlock} \\
            \end{eqnarray*}
        \item Durch einen Lock-Mechanismus (z.B. Semaphoren), kann garantiert wreden, dass nur ein Auto exklusiven Zugriff auf die Straße bekommt.
    \end{enumerate}
    \section{Semaphore und aktives Warten (Spinlock)}
    \begin{enumerate}[label=(\alph*)]
        \item
            Vorteil Spinlock: Prozess kann beim Aufheben der Sperre sofort weiterlaufen, ohne auf erst vom Scheduler aufgerufen werden zu müssen. Beim Semaphore kann es einen Moment dauern, bis der Prozess wieder an der Reihe ist.

            Vorteil Semaphore: Scheduler kann während des wartens andere Prozesse ausführen, es wird keine CPU-Zeit verschwendet, im Gegensatz zum aktiven Warten.
        \item Nein, da ein Spinlock den Prozessor blockiert, kann der Prozess, der sich im gesperrten Abschnitt befindet, nicht weiter ausgeführt werden. Der Abschnitt bleibt also ewig gesperrt.
        \item Ja, müssen sie, sonst können sich weiterhin mehrere Prozesse gleichzeitig (auf Mehrprozessorsystemen) im kritischen Abschnitt befinden.
    \end{enumerate}
\end{document}
