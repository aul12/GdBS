\documentclass[DIN, pagenumber=false, fontsize=11pt, parskip=half]{scrartcl}

\usepackage{amsmath}
\usepackage{amsfonts}
\usepackage{amssymb}
\usepackage{enumitem}
\usepackage[utf8]{inputenc} % this is needed for umlauts
\usepackage[ngerman]{babel} % this is needed for umlauts
\usepackage[T1]{fontenc} 
\usepackage{commath}
\usepackage{xcolor}
\usepackage{booktabs}
\usepackage{float}
\usepackage{tikz-timing}
\usepackage{tikz}
\usepackage{multirow}

\usetikzlibrary{calc,shapes.multipart,chains,arrows}

\title{Grundlagen der Betriebssysteme}
\author{Tim Luchterhand, Paul Nykiel (Gruppe 017)}

\begin{document}
    \maketitle
    \section{Fahrstuhl-Scheduling}
    Es wird angenommen, dass der Fahrstuhl nach jedem Paket in dem Stockwerk verbleibt.
    \begin{enumerate}[label=(\alph*)]
            \item 
                Paket 1: \{5,3,7\} 

                Paket 2: \{6,9,2\} 

                Paket 3: \{1,4,8\} 

                Zurückgelegte Strecke: 26 Stockwerke
            \item 
                Paket 1: \{5,7,3\} 

                Paket 2: \{2,6,9\} 

                Paket 3: \{8,4,1\} 

                Zurückgelegte Strecke: 23 Stockwerke

            \item 
                \{5,7,9,6,4,3,2,1,8\}

                Zurückgelegte Strecke: 20 Stockwerke
    \end{enumerate}

    \section{Gemischte Rückblicke}
    \begin{enumerate}[label=(\alph*)]
        \item Es sind 8, 16, 24 und 32 Bit breite Zeichen möglich
        \item Read, Write, Execute
        \item Access-Control-List
        \item 
            \begin{eqnarray*}
                -{(15)}_{16} = {256}_{10} - {21}_{10} = {235}_{10} = {1110\ 1011}_{2}
            \end{eqnarray*}
        \item Bei Algorithmen, die die FIFO-Anomalie aufweisen, kann es sein, dass trotz mehr verfügbaren Speicherkacheln mehr Seiten ein- und auslagerungen 
            notwendig sind, als bei weniger verfügbaren Speicherkacheln.
        \item
            ${0000}_4$ und ${1000}_4$
    \end{enumerate}
\end{document}
