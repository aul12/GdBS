\documentclass[DIN, pagenumber=false, fontsize=11pt, parskip=half]{scrartcl}

\usepackage{amsmath}
\usepackage{amsfonts}
\usepackage{amssymb}
\usepackage{enumitem}
\usepackage[utf8]{inputenc} % this is needed for umlauts
\usepackage[ngerman]{babel} % this is needed for umlauts
\usepackage[T1]{fontenc} 
\usepackage{commath}
\usepackage{xcolor}
\usepackage{booktabs}
\usepackage{float}
\usepackage{tikz-timing}
\usepackage{tikz}

\title{Grundlagen der Betriebssysteme}
\author{Tim Luchterhand, Paul Nykiel (Gruppe 017)}

\begin{document}
    \maketitle
    \section{FAT}
    \begin{enumerate}[label=(\alph*)]
        \item In der zweiten Zeile stehen die Informationen über das jeweilige Cluster (frei, beschädigt, belegt mit Addresse auf das nächste Cluster)
        \item 
            \begin{enumerate}
                \item ${02}_{16}-{04}_{16}-{03}_{16}$
                \item ${05}_{16}-{06}_{16}-{09}_{16}$ 
                \item ${07}_{16}$
                \item ${0B}_{16}-{0F}_{16}$
                \item ${0C}_{16}-{0D}_{16}-{0E}_{16}-{10}_{16}-{11}_{16}$            
            \end{enumerate}
        \item 
            \begin{enumerate}
                \item Datei 1 korrespondiert zu c) (1 Cluster). 
                    
                    Minimale Größe: 1 Byte, Maximale Größe 8KB
                \item Datei 2 korrespondiert zu a) oder b) (3 Cluster). 
                    
                    Minimale Größe: 16KB + 1 Byte, Maximale Größe 24KB
                \item Datei 3 korrespondiert zu a) oder b) (3 Cluster). 
                    
                    Minimale Größe: 16KB + 1 Byte, Maximale Größe 24KB
                \item Datei 4 korrespondiert zu d) (2 Cluster). 
                    
                    Minimale Größe: 8KB + 1 Byte, Maximale Größe 16KB
                \item Datei 5 korrespondiert zu e) (5 Cluster). 
                    
                    Minimale Größe: 40KB + 1 Byte, Maximale Größe 48KB

                    Die Datei braucht eigentlich 6 Blöcke, eventuell gehört der beschädigte Block (${12}_{16}$) zur Datei dazu (als erster Block).
            \end{enumerate}
    \end{enumerate}

    \section{NTFS}
    Es wird angenommen, dass für die Datei rsa.key zwei Cluster direkt hintereinander im Speicher frei sind.
    \begin{table}[H]
        \centering
        \begin{tabular}{p{.33\textwidth}|p{.12\textwidth}|p{.15\textwidth}|p{.2\textwidth}}
            \toprule
            Standardinfo & Dateiname & Zugriffsrechte & Daten \\
            \midrule
            Länge: 100 Bytes,\\ \textit{MS-DOS Attribute, Zeitstempel, Anzahl der Hardlinks, Sequenznummer der gültigen File Reference} & hallo.txt & \textit{Zugriffsrechte} & Daten\\
            \midrule
            Länge: 8 KBytes,\\ \textit{MS-DOS Attribute, Zeitstempel, Anzahl der Hardlinks, Sequenznummer der gültigen File Reference} & rsa.key& \textit{Zugriffsrechte} & VCN: 0 \textit{LCN} Anzahl Cluster: 2\\
            \midrule
            Länge: 44 KBytes,\\ \textit{MS-DOS Attribute, Zeitstempel, Anzahl der Hardlinks, Sequenznummer der gültigen File Reference} & cat.jpg& \textit{Zugriffsrechte} & VCN: 0 LCN: 17 Anzahl Cluster: 1\newline
                VCN: 1 LCN: 19 Anzahl Cluster: 1\newline
                VCN: 2 LCN: 22 Anzahl Cluster: 9\\
            \bottomrule
        \end{tabular}
    \end{table}
\end{document}
