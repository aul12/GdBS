\documentclass[DIN, pagenumber=false, fontsize=11pt, parskip=half]{scrartcl}

\usepackage{amsmath}
\usepackage{amsfonts}
\usepackage{amssymb}
\usepackage{enumitem}
\usepackage[utf8]{inputenc} % this is needed for umlauts
\usepackage[ngerman]{babel} % this is needed for umlauts
\usepackage[T1]{fontenc} 
\usepackage{commath}
\usepackage{xcolor}
\usepackage{booktabs}
\usepackage{float}
\usepackage{tikz-timing}

\title{Grundlagen der Betriebssysteme}
\author{Tim Luchterhand, Paul Nykiel (Gruppe 017)}

\begin{document}
    \maketitle
    \section{Prozesszustände}
    \begin{enumerate}[label=(\alph*)]
        \item Prozesse können sich in den folgenden Zuständen befinden:
            \begin{enumerate}
                \item Starten
                \item Bereit
                \item Laufend
                \item Wartend
                \item Gestoppt
            \end{enumerate}
        \item $ $
            \begin{table}[H]
                \centering
                \begin{tabular}{c|ccccc}
                    \toprule 
                    Starten & Bereit & Laufend & Wartend & Gestoppt \\
                    \midrule
                    Starten \\
                    Bereit & Prozess fertig initialisiert \\
                    Laufend & - & Prozess wird vom Scheduler ausgewählt \\
                    Wartend & \\
                    Gestoppt & \\
                    \bottomrule
                \end{tabular}
                \caption{Mögliche Zustandsübergänge}
            \end{table}
    \end{enumerate}
    \section{Shortest Job First Scheduling}
    \begin{enumerate}[label=(\alph*)]
        \item Bei shortest Job First werden viele Prozesse in kurzer Zeit abgearbeitet, da die einzelnen Prozesse jeweils nur kurz laufen.
        \item Prozesse die lange laufen werden gar nicht oder nur sehr spät ausgeführt.
        \item Beim präemptiven shortest Job First Scheduling wird auch bei einem \glqq{}freiwilligen\grqq{} Aufrufen des Schedulers (z.B. durch einen Syscall), der Scheduling-Algorithmus aufgerufen und wartende Prozesse die kürzer brauchen ausgeführt.

            Beim nicht präemptiven Shortest Job First Scheduling läuft ein Prozess ohne Unterbrechung. Der Prozess wird nur unterbrochen wenn er wartet oder stoppt.
        \item $ $
            \begin{figure}[H]
                \centering
                \begin{tikztimingtable}
                    \begin{scope}
                        \horlines{}
                        \foreach \x in {1,...,6}
                            \draw (\x,1) -- (\x, 4)
                    \end{scope}
                    Prozess A & LL H  \\
                    Prozess B & LL H  \\
                    Prozess C & LL H  \\
                \end{tikztimingtable}
            \end{figure}
    \end{enumerate}
\end{document}
