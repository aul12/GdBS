\documentclass[DIN, pagenumber=false, fontsize=11pt, parskip=half]{scrartcl}

\usepackage{amsmath}
\usepackage{amsfonts}
\usepackage{amssymb}
\usepackage{enumitem}
\usepackage[utf8]{inputenc} % this is needed for umlauts
\usepackage[ngerman]{babel} % this is needed for umlauts
\usepackage[T1]{fontenc} 
\usepackage{commath}
\usepackage{xcolor}
\usepackage{booktabs}
\usepackage{float}
\usepackage{tikz-timing}
\usepackage{tikz}

\title{Grundlagen der Betriebssysteme}
\author{Tim Luchterhand, Paul Nykiel (Gruppe 017)}

\begin{document}
    \maketitle
    \section{Round-Robin und Highest Priority Scheduling}
    Ein High-Sigal bedeutet, dass der jeweilige Prozess gerade läuft. Ein Low-Signal bedeutet, dass der Prozess bereit ist.
    \begin{enumerate}[label=(\alph*)]
        \item $ $
            \begin{figure}[H]
                \centering
                \begin{tikztimingtable}[timing/slope=0, timing/wscale=4.0]
                    Prozess A & [L] hh lh hl ll lh hh hl s\\
                    Prozess B &     sl hl ll lh ls ss ss s\\
                    Prozess C &     ss sl lh hl hl ss ss s\\
                    \begin{extracode}
                        \vertlines[help lines]{0, 4, ..., \twidth}
                        \foreach \x in {0, ...,7}
                            \node at (\x*4,2.5) {\x s};
                    \end{extracode}
                \end{tikztimingtable}
                \caption{Round-Robin Scheduling}
            \end{figure}
        \item $ $
            \begin{figure}[H]
                \centering
                \begin{tikztimingtable}[timing/slope=0, timing/wscale=4.0]
                    Prozess A & [L] hh hh ll ll lh hh hl s \\
                    Prozess B &     sl ll hl ll ll ll lh l \\
                    Prozess C &     ss sl lh hh hl ss ss s \\
                    \begin{extracode}
                        \vertlines[help lines]{0, 4, ..., \twidth}
                        \foreach \x in {0, ...,7}
                            \node at (\x*4,2.5) {\x s};
                    \end{extracode}
                \end{tikztimingtable}
                \caption{Highest Priority First Scheduling (nicht präemptiv)}
            \end{figure}
    \end{enumerate}
    \section{Threads und Prozesse}
            \begin{enumerate}[label=(\alph*)]
                \item 
                    \begin{enumerate}
                        \item Daten (Variablen \ldots)
                        \item Datein (Dateideskriptoren)
                    \end{enumerate}
                \item
                    \begin{enumerate}
                        \item Rechenzeit
                    \end{enumerate}
                \item Nachteile:
                    \begin{enumerate}
                        \item Blockierte User Threads blockieren alle Kernel-Threads
                        \item Echte Parallelität erfordert wiederrum mehrere Kernel-Threads
                    \end{enumerate}
                    Da User-Level-Threads ohne Kernel-Threads keine Parallelität bereitstellen braucht es trotzdem Kernel-Threads um Vorteile zu erreichen, damit kann man allerdings auch direkt Kernel-Threads verwenden.
            \end{enumerate}
\end{document}
