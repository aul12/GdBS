\documentclass[DIN, pagenumber=false, fontsize=11pt, parskip=half]{scrartcl}

\usepackage{amsmath}
\usepackage{amsfonts}
\usepackage{amssymb}
\usepackage{enumitem}
\usepackage[utf8]{inputenc} % this is needed for umlauts
\usepackage[ngerman]{babel} % this is needed for umlauts
\usepackage[T1]{fontenc} 
\usepackage{commath}
\usepackage{xcolor}
\usepackage{booktabs}
\usepackage{float}
\usepackage{tikz-timing}
\usepackage{tikz}
\usepackage{multirow}

\usetikzlibrary{calc,shapes.multipart,chains,arrows}

\title{Grundlagen der Betriebssysteme}
\author{Tim Luchterhand, Paul Nykiel (Gruppe 017)}

\begin{document}
    \maketitle
    \section{Seitenadressierung}
    \begin{enumerate}
        \item Software: Addition des Segmenttabellenbasisregisters und der logischen Segmentnummer der logischen Adresse
            ergibt Adresse des Segmenteintrags.
        \item Software: Lesen des zugehörigen Segmenteintrags, der die Startadresse der zugehörigen Seiten-Kachel-Tabelle
            (SKT) enthält.
        \item Software: Addition der Startadresse der SKT und der logischen SKT-Nummer der logischen Adresse ergibt die Addresse des SKT-Eintrags.
        \item Software: Vergleich der Seitennummer mit Segmentlänge. Falls die Seitennummer außerhalb des Segments: Unterbrechung
        \item Hardware: Laden des SKT-Eintrags
        \item Hardware: Ermitteln des Präsenzbits. Dies ist in diesem Fall: null (0)
        \item Hardware: Untebrechung (Page fault).
        \item Software: Blockieren des Prozesses, einlagern der benötigten Seite in die freie Kachel.
        \item Hardware: Kacheladresse in passendem SKT-Eintrag anpassen und Präsenzbit auf eins (1) setzten.
        \item Software: Prozess aufwecken und Speicherzugriff wiederholen.
    \end{enumerate}

    \section{Ersetzungsstrategien}
    \begin{table}[H]
        \centering
        \begin{tabular}{lc|c|c|c|c|c|c|c|c|c|c}
            \toprule
            Referenzfolge & & 1 & 2 & 3 & 1 & 2 & 4 & 5 & 1 & 2 & 3 \\
            \midrule
            \multirow{3}{*}{Hauptspeicher} & Kachel 1  & \textbf{1} & 1 & 1 & 1 & 1 & 1 & \textbf{5} & 5 & 5 & \textbf{3}\\
            & Kachel 2 & - & \textbf{2} & 2 & 2 & 2 & 2 & 2 & \textbf{1} & 1 & 1\\
            & Kachel 3 & - & - & \textbf{3} & 3 & 3 & \textbf{4} & 4 & 4 & \textbf{2} & 2\\
            \midrule
            \multirow{3}{3cm}{Kontrollzustände / Referenzbits} & Kachel 1 & 0 & 1 & 2 & 0 & 1 & 2 & 0 & 1 & 2 & 0\\
            & Kachel 2 & - & 0 & 1 & 2 & 0 & 1 & 2 & 0 & 1 & 2\\
            & Kachel 3 & - & - & 0 & 1 & 2 & 0 & 1 & 2 & 0 & 1\\
            \bottomrule
        \end{tabular}
        \caption{Least Recently Used; Seitenersetzungen: 8}
    \end{table}
    \begin{table}[H]
        \centering
        \begin{tabular}{lc|c|c|c|c|c|c|c|c|c|c}
            \toprule
            Referenzfolge & & 1 & 2 & 3 & 1 & 2 & 4 & 5 & 1 & 2 & 3 \\
            \midrule
            \multirow{3}{*}{Hauptspeicher} & Kachel 1  & \textbf{1} & 1 & 1 & 1 & 1 & \textbf{4} & 4 & 4 & \textbf{2} & 2\\
            & Kachel 2 & - & \textbf{2} & 2 & 2 & 2 & 2 & \textbf{5} & 5 & 5 & \textbf{3}\\
            & Kachel 3 & - & - & \textbf{3} & 3 & 3 & 3 & 3 & \textbf{1} & 1 & 1\\
            \midrule
            \multirow{3}{3cm}{Kontrollzustände / Referenzbits} & Kachel 1 & 0 & 1 & 2 & 3 & 4 & 0 & 1 & 2 & 0 & 1\\
            & Kachel 2 & - & 0 & 1 & 2 & 3 & 4 & 0 & 1 & 2 & 0\\
            & Kachel 3 & - & - & 0 & 1 & 2 & 3 & 4 & 0 & 1 & 2\\
            \bottomrule
        \end{tabular}
        \caption{First-In First-Out; Seitenersetztungen: 8}
    \end{table}
    \begin{table}[H]
        \centering
        \begin{tabular}{lc|c|c|c|c|c|c|c|c|c|c}
            \toprule
            Referenzfolge & & 1 & 2 & 3 & 1 & 2 & 4 & 5 & 1 & 2 & 3 \\
            \midrule
            \multirow{3}{*}{Hauptspeicher} & Kachel 1 & \textbf{1} & 1 & 1 & 1 & 1 & \textbf{4} & 4 & 4 & \textbf{2} & 2\\
            & Kachel 2 & - & \textbf{2} & 2 & 2 & 2 & 2 & \textbf{5} & 5 & 5 & \textbf{3}\\
            & Kachel 3 & - & - & \textbf{3} & 3 & 3 & 3 & 3 & \textbf{1} & 1 & 1\\
            \midrule
            \multirow{4}{3cm}{Kontrollzustände / Referenzbits} & Kachel 1 & 1 & 1 & 1 & 1 & 1 & 1 & 1 & 1 & 1 & 1\\
            & Kachel 2 & 0 & 1 & 1 & 1 & 1 & 0 & 1 & 1 & 0 & 1\\
            & Kachel 3 & 0 & 0 & 1 & 1 & 1 & 0 & 0 & 1 & 0 & 0\\
            & Umlaufzeiger & 2 & 3 & 1 & 1 & 1 & 2 & 3 & 1 & 2 & 3\\
            \bottomrule
        \end{tabular}
        \caption{Second chance, clock; Seitenersetzungen: 8}
    \end{table}
\end{document}
